\documentclass[conference]{IEEEtran}
\usepackage{polski}
\usepackage[utf8]{inputenc}
\usepackage{blindtext, graphicx}

\ifCLASSINFOpdf
  % \usepackage[pdftex]{graphicx}
  % declare the path(s) where your graphic files are
  % \graphicspath{{../pdf/}{../jpeg/}}
  % and their extensions so you won't have to specify these with
  % every instance of \includegraphics
  % \DeclareGraphicsExtensions{.pdf,.jpeg,.png}
\else
  % or other class option (dvipsone, dvipdf, if not using dvips). graphicx
  % will default to the driver specified in the system graphics.cfg if no
  % driver is specified.
  % \usepackage[dvips]{graphicx}
  % declare the path(s) where your graphic files are
  % \graphicspath{{../eps/}}
  % and their extensions so you won't have to specify these with
  % every instance of \includegraphics
  % \DeclareGraphicsExtensions{.eps}
\fi

% Graphics:
% \graphicspath{{folder/}{otherfolder/}}
\graphicspath{{ObrazAnkieta/}}
% correct bad hyphenation here
\hyphenation{op-tical net-works semi-conduc-tor}


\begin{document}
%
% paper title
% can use linebreaks \\ within to get better formatting as desired
\title{Obserwacje grup szkolnych w Centrum Nauki Kopernik -- raport}


% author names and affiliations
% use a multiple column layout for up to three different
% affiliations
\author{\IEEEauthorblockN{Ahmed Abdelkarim, Aleksandra Hernik, Ada Wrońska}
\IEEEauthorblockA{Wydział Matematyki i Nauk Informacyjnych\\
Politechnika Warszawska}}

\maketitle


\begin{abstract}
%\boldmath
Celem tego raportu jest przedstawienie wyników analizy danych dotyczących sposobu zwiedzania i charakterystyki dwunastoletnich dzieci w Centrum Nauki Kopernik. Dane składały się z dwóch tabel -- ankiety, która była wypełniana przez dzieci i zawierała informacje na ich temat, które zostały podsumowane jako \textit{kapitał naukowy}, oraz obserwacji dzieci, z takimi informacjami jak czas spędzony przy każdym eksponacie i stopień poświęconej mu uwagi, a także listą innych dzieci biorących udział w interakcji. Przeprowadzoną analizę danych można podzielić na poszukiwanie zależności między pewnymi cechami dzieci (i ich środowiska), znajdowanie prawidłowości w sposobie zwiedzania dzieci, i stworzenie charakterystyki eksponatów.
\end{abstract}
\IEEEpeerreviewmaketitle



\section{Wstęp}
\subsection{Opis danych}
Opisywany projekt polegał na przeanalizowaniu danych z wizyt dwunastoletnich dzieci w Centrum Nauki Kopernik. Dostępne dane znajdowały się w dwóch tabelach:
\begin{itemize}
\item \textit{Ankieta.csv} (237 wierszy), składająca się z: \begin{itemize}
\item ID (numer klasy i ucznia),
\item Informacja, czy uczeń był obserwowany,
\item Płeć,
\item Ukończenie studiów przez rodziców,
\item Praca rodziców,
\item Oceny z matematyki, przyrody i języka polskiego,
\item Wymarzony zawód,
\item Jego pokrewieństwo z nauką,
\item Odpowiedzi na poszczególne pytania z ankiety (poziom kontaktu z nauką).
\end{itemize}
\item \textit{Obserwacje.csv} (15225 wierszy), składająca się z: \begin{itemize}
\item ID ucznia,
\item Nazwa eksponatu,
\item Galeria,
\item Kategoria zdarzenia (przerwa, eksponat i inne),
\item Czas rozpoczęcia, trwania i zakończenia zdarzenia,
\item Osoby towarzyszące w zdarzeniu,
\item Poziom eksploracji eksponatu (patrzenie, dotykanie, używanie lub eksperymentowanie)
\item Informacja o tym, czy dziecko przeczytało opis,
\item Informacja o tym, czy dziecko rozmawiało z animatorem,
\item Uwagi na temat zdarzenia.
\end{itemize}
\end{itemize}
Przed rozpoczęciem analizy dane zostały do niej przygotowane poprzez:
\begin{itemize}
\item Usunięcie brakujących danych (np. nieczytelne pismo jako wymarzona praca),
\item Posortowanie odpowiedzi w ankiecie (factor w~języku R),
\item Poprawienie literówek i usunięcie nadmiarowych spacji w nazwach eksponatów,
\item Usunięcie nieznanych osób towarzyszących.
\end{itemize}
Wstępna analiza danych wykazała, że liczba obserwowanych dzieci, dla których są dane z~ankiety, była bardzo ograniczona (jedynie 79 dzieci). Z~tego powodu konieczne było zrezygnowanie z~niektórych kierunków badań, szczególnie tych, które dotyczyły analizy sposobu zwiedzania przez grupki dzieci -- w~zdecydowanej większości z~nich znajdowały się dzieci, które nie były obserwowane lub nie wypełniały ankiety.
\subsection{Kierunki badań}
Nasza analiza skupiła się na następujących kierunkach badań:
\begin{itemize}
\item Poszukiwanie zależności między różnymi cechami dziecka i jego środowiska,
\item Analiza sposobu zwiedzania przez dzieci, w szczególności obserwacja tworzących się grupek,
\item Analiza eksponatów w Centrum Nauki Kopernik,
\item Wykorzystanie klasteryzacji w celu pogrupowania dzieci i eksponatów.
\end{itemize}

\section{Cechy dzieci i czynniki środowiskowe}
W celu poszukiwania zależności między cechami dziecka, a jego środowiskiem skupiliśmy się na analizie danych zawartych w Ankiecie. Dane te zostały zebrane za pomocą odpowiedzi zebranych z ankiet wypełnionych przez dzieci. W ramach wstępnej obróbki danych zastąpiliśmy zwoty takie jak "Nie wiem", "Nie wie" itp. standardową formą zapisu braku danych (NA). To bardzo ułatwiło analizę danych dotyczących wiedzy dzieci na temat ich własnych rodziców.

Pierwszym zadaniem było spojrzenie na studia i pracę rodziców oczami dzieci. Chcieliśmy sprawdzić zależność między płcią dziecka, a jego wiedzą na temat rodziców. W tym celu wybraliśmy dane w oparciu na płeć i później pogrupowaliśmy ze względu na wiedzę. Podzieliliśmy to na trzy kategorie:

\begin{itemize}
	\item Tak - dziecko ma wiedzę o obojgu rodzicach,
	\item O jednym - dziecko ma wiedzę o jednym z rodziców,
	\item Nie - dziecko nie wie nic o rodzicach.
\end{itemize}

W ten sposób otrzymaliśmy poniższe wyniki:

\begin{figure}
	\centering
	\includegraphics[width=0.5\textwidth]{1.png}
	\caption{Graf dotyczący wiedzy dzieci o studiach rodziców.}
	\label{fig:studies_knowledge}
\end{figure}
\begin{figure}
	\centering
	\includegraphics[width=0.5\textwidth]{2.png}
	\caption{Graf dotyczący wiedzy dzieci o pracy rodziców.}
	\label{fig:work_knowledge}
\end{figure}

Jak widać na załączonych rysunkach (ref. Rysunek~\ref{fig:studies_knowledge} oraz Rysunek~\ref{fig:work_knowledge}) dziewczynki mają lepszą wiedzę o rodzicach. Dodatkowo okazuje się, że dzieci więcej wiedzą o pracy rodziców, niż o ich studiach.

Kolejną z rzeczy, które chcieliśmy sprawdzić była zależność między otoczeniem dziecka, a jego ocenami. W związku z tym chcieliśmy sprawdzić zależność między ilością książek, studiami lub dopingiem rodziców. W tym celu policzyliśmy dla każdego dziecka średnią ocen.

\begin{figure}
	\centering
	\includegraphics[width=0.5\textwidth]{3.png}
	\caption{Graf dotyczący zależności między dopingiem rodziców a  średnią ocen.}
	\label{fig:doping_oceny}
\end{figure}

Jak widać na załączonym grafie (ref. Rysunek~\ref{fig:doping_oceny}), doping rodziców zdaje się nie mieć wpływu na oceny uczniów. Niestety może to być związane z brakiem zrozumienia pytania przez dzieci.

\begin{figure}
	\centering
	\includegraphics[width=0.5\textwidth]{4.png}
	\caption{Graf dotyczący zależności między studiami rodziców a średnią ocen.}
	\label{fig:studia_oceny}
\end{figure}

Studia rodziców zdają się mieć trochę większy wpływ na wyniki dzieci (ref. Rysunek~\ref{fig:studia_oceny}). Dzieci, których rodzice ukończyli studia i dzieci o tym wiedzą mają średnio lepsze wyniki w nauce.

\begin{figure}
	\centering
	\includegraphics[width=0.5\textwidth]{5.png}
	\caption{Graf dotyczący zależności między ilością książek w domu a średnią ocen.}
	\label{fig:ksiazki_oceny}
\end{figure}

Najlepsze efekty dało porównanie ilości książek w domu ze średnią (ref. Rysunek~\ref{fig:ksiazki_oceny}). Wyraźnie lepsze wyniki osiągają dzieci, w których otoczeniu znajduje się wiele książek. Jeśli w otoczeniu dziecka znajduje się przynajmniej 50 książek, to przeciętnie średnia ocen dziecka jest wyższa niż 4.0, co jest naprawdę bardzo dobrym wynikiem. Natomiast gdy nie ma żadnej książki, średnia spada poniżej 3.0. Jasno z tego wynika, że książki wpływają pozytywnie na rozwój dziecka.

W ramach kolejnego zadania, chcieliśmy sprawdzić zależność między obliczonym Kapitałem Naukowym, a studiami rodziców i ocenami dzieci, jako że okazały się byc one najbardziej miarodajne.

\begin{figure}
	\centering
	\includegraphics[width=0.5\textwidth]{6.png}
	\caption{Graf dotyczący zależności między studiami rodziców a kapitałem naukowym.}
	\label{fig:studia_kapital}
\end{figure}
\begin{figure}
	\centering
	\includegraphics[width=0.5\textwidth]{7.png}
	\caption{Graf dotyczący zależności między ocenami a kapitałem naukowym z odliczonymi ocenami.}
	\label{fig:oceny_kapital}
\end{figure}

Wzięcie pod uwagę Kapitału Naukowego (ref. Rysunek~\ref{fig:studia_kapital}) zamiast ocen (ref. Rysunek~\ref{fig:studia_oceny}) dało bardziej zrównane wyniki. Co może świadczyć o niedokładności Kapitału Naukowego lub o braku zależności między ocenami a studiami rodziców.

Bardzo ładną natomiast okazała się zależność między ocenami, a Kapitałem Naukowym liczonym bez ocen (ref. Rysunek~\ref{fig:oceny_kapital}). Jest to prawie zależność liniowa, z niewielkimi odchyłami. Sugeruje ona, że Kapitał Naukowy jest dość dobrym przybliżeniem potencjału dziecka.

Wiele z danych, w tym na przykład zawody, zawierało literówki, bądź też bardzo różne formy opisu tego samego. W celu przefiltrowania danych pogrupowaliśmy zawody na związane z konkretnymi przedmiotami: Matematyką, Polskim i Przyrodą.
Chcieliśmy to także powiązać z ocenami z konkretnych przedmiotów. W tym celu pogrupowaliśmy oceny wedle przedmiotów by zobaczyć czy istnieje jakaś zależność.

\begin{figure}
	\centering
	\includegraphics[width=0.5\textwidth]{8.png}
	\caption{Graf dotyczący ocen z konkretnych przedmiotów.}
	\label{fig:oceny}
\end{figure}

Jak widać na załączonym boxplocie (ref. Rysunek~\ref{fig:oceny}) oceny są bardzo wyrównane, choć matematyka zdaje się ogółem sprawiać więcej problemów.

Do przetwarzania ocen wykorzystaliśmy tylko oceny dzieci, których rodzice mieli zawód związany z Matematyką, Polskim i Przyrodą, natomiast odrzuciliśmy Inne i NA. Stworzyliśmy trzy grupy wedle przedmiotów i w nich podzieliliśmy dane na związane z przedmiotem (uczeń miał rodzica związanego z przedmiotem gdy przynajmniej jeden rodzic pracował \"w przedmiocie\"), bądź też nie. Wewnątrz tych podgrup wyliczyliśmy średnią ocen i porównaliśmy.

\begin{figure}
	\centering
	\includegraphics[width=0.5\textwidth]{9.png}
	\caption{Graf dotyczący zależności między pracą rodziców a oceną z danego przedmiotu.}
	\label{fig:oceny_praca}
\end{figure}

Jak widać (ref. Rysunek~\ref{fig:oceny_praca}), uczniowie najwięcej mogą zyskać, gdy praca ich rodziców jest powiązana z językiem polskim - różnica ocen między dziećmi osób niezwiązanych z polskim była największa. Zaskakująco najmniejsza była związana z matematyką.

\begin{figure}
	\centering
	\includegraphics[width=0.5\textwidth]{10.png}
	\caption{Graf dotyczący zależności między wymarzoną pracą a oceną z danego przedmiotu.}
	\label{fig:oceny_wymarzona}
\end{figure}

Bardzo podobne wyniki osiągnęliśmy przy porównaniu ocen z wymarzoną pracą (ref. Rysunek~\ref{fig:oceny_wymarzona}). W tym przypadku jednak była większa różnica między ocenami dzieci "Do zawodu", a ocenami "Poza zawodem". Uznaliśmy, że może to wynikać z tego, że przedmiot typu matematyka jest uznawany za bardziej ustruktyruzowany, a co za tym idzie powtarzalny, w przeciwieństwie do polskiego, który jest bardziej oparty na kreatywności, której nie da się wpisać w schemat.

\section{Charakterystyka sposobu zwiedzania przez dzieci}

\section{Analiza eksponatów}

\section{Wnioski}








\end{document}


