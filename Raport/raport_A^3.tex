\documentclass[conference]{IEEEtran}
\usepackage{polski}
\usepackage[utf8]{inputenc}
\usepackage{blindtext, graphicx}

\ifCLASSINFOpdf
  % \usepackage[pdftex]{graphicx}
  % declare the path(s) where your graphic files are
  % \graphicspath{{../pdf/}{../jpeg/}}
  % and their extensions so you won't have to specify these with
  % every instance of \includegraphics
  % \DeclareGraphicsExtensions{.pdf,.jpeg,.png}
\else
  % or other class option (dvipsone, dvipdf, if not using dvips). graphicx
  % will default to the driver specified in the system graphics.cfg if no
  % driver is specified.
  % \usepackage[dvips]{graphicx}
  % declare the path(s) where your graphic files are
  % \graphicspath{{../eps/}}
  % and their extensions so you won't have to specify these with
  % every instance of \includegraphics
  % \DeclareGraphicsExtensions{.eps}
\fi


% correct bad hyphenation here
\hyphenation{op-tical net-works semi-conduc-tor}


\begin{document}
%
% paper title
% can use linebreaks \\ within to get better formatting as desired
\title{Obserwacje grup szkolnych w Centrum Nauki Kopernik -- raport}


% author names and affiliations
% use a multiple column layout for up to three different
% affiliations
\author{\IEEEauthorblockN{Ahmed Abdelkarim, Aleksandra Hernik, Ada Wrońska}
\IEEEauthorblockA{Wydział Matematyki i Nauk Informacyjnych\\
Politechnika Warszawska}}

\maketitle


\begin{abstract}
%\boldmath
Celem tego raportu jest przedstawienie wyników analizy danych dotyczących sposobu zwiedzania i charakterystyki dwunastoletnich dzieci w Centrum Nauki Kopernik. Dane składały się z dwóch tabel -- ankiety, która była wypełniana przez dzieci i zawierała informacje na ich temat, które zostały podsumowane jako \textit{kapitał naukowy}, oraz obserwacji dzieci, z takimi informacjami jak czas spędzony przy każdym eksponacie i stopień poświęconej mu uwagi, a także listą innych dzieci biorących udział w interakcji. Przeprowadzoną analizę danych można podzielić na poszukiwanie zależności między pewnymi cechami dzieci (i ich środowiska), znajdowanie prawidłowości w sposobie zwiedzania dzieci, i stworzenie charakterystyki eksponatów.
\end{abstract}
\IEEEpeerreviewmaketitle



\section{Wstęp}
\subsection{Opis danych}
Opisywany projekt polegał na przeanalizowaniu danych z wizyt dwunastoletnich dzieci w Centrum Nauki Kopernik. Dostępne dane znajdowały się w dwóch tabelach:
\begin{itemize}
\item \textit{Ankieta.csv} (237 wierszy), składająca się z: \begin{itemize}
\item ID (numer klasy i ucznia),
\item Informacja, czy uczeń był obserwowany,
\item Płeć,
\item Ukończenie studiów przez rodziców,
\item Praca rodziców,
\item Oceny z matematyki, przyrody i języka polskiego,
\item Wymarzony zawód,
\item Jego pokrewieństwo z nauką,
\item Odpowiedzi na poszczególne pytania z ankiety (poziom kontaktu z nauką).
\end{itemize}
\item \textit{Obserwacje.csv} (15225 wierszy), składająca się z: \begin{itemize}
\item ID ucznia,
\item Nazwa eksponatu,
\item Galeria,
\item Kategoria zdarzenia (przerwa, eksponat i inne),
\item Czas rozpoczęcia, trwania i zakończenia zdarzenia,
\item Osoby towarzyszące w zdarzeniu,
\item Poziom eksploracji eksponatu (patrzenie, dotykanie, używanie lub eksperymentowanie)
\item Informacja o tym, czy dziecko przeczytało opis,
\item Informacja o tym, czy dziecko rozmawiało z animatorem,
\item Uwagi na temat zdarzenia.
\end{itemize}
\end{itemize}
Przed rozpoczęciem analizy dane zostały do niej przygotowane poprzez:
\begin{itemize}
\item Usunięcie brakujących danych (np. nieczytelne pismo jako wymarzona praca),
\item Posortowanie odpowiedzi w ankiecie (factor w~języku R),
\item Poprawienie literówek i usunięcie nadmiarowych spacji w nazwach eksponatów,
\item Usunięcie nieznanych osób towarzyszących.
\end{itemize}
Wstępna analiza wykazała, że liczba obserwowanych dzieci, dla których są dane z~ankiety, była bardzo ograniczona (jedynie 79 dzieci). Z~tego powodu konieczne było zrezygnowanie z~niektórych kierunków badań, szczególnie tych, które dotyczyły analizy sposobu zwiedzania przez grupki dzieci -- w~zdecydowanej większości z~nich znajdowały się dzieci, które nie były obserwowane lub nie wypełniały ankiety.
\subsection{Subsection Heading Here}
\blindtext




\section{Conclusion}
\blindtext





% if have a single appendix:
%\appendix[Proof of the Zonklar Equations]
% or
%\appendix  % for no appendix heading
% do not use \section anymore after \appendix, only \section*
% is possibly needed

% use appendices with more than one appendix
% then use \section to start each appendix
% you must declare a \section before using any
% \subsection or using \label (\appendices by itself
% starts a section numbered zero.)
%


\appendices
\section{Proof of the First Zonklar Equation}
\blindtext

% use section* for acknowledgement
\section*{Acknowledgment}


The authors would like to thank...


% Can use something like this to put references on a page
% by themselves when using endfloat and the captionsoff option.
\ifCLASSOPTIONcaptionsoff
  \newpage
\fi



% trigger a \newpage just before the given reference
% number - used to balance the columns on the last page
% adjust value as needed - may need to be readjusted if
% the document is modified later
%\IEEEtriggeratref{8}
% The "triggered" command can be changed if desired:
%\IEEEtriggercmd{\enlargethispage{-5in}}

% references section

% can use a bibliography generated by BibTeX as a .bbl file
% BibTeX documentation can be easily obtained at:
% http://www.ctan.org/tex-archive/biblio/bibtex/contrib/doc/
% The IEEEtran BibTeX style support page is at:
% http://www.michaelshell.org/tex/ieeetran/bibtex/
%\bibliographystyle{IEEEtran}
% argument is your BibTeX string definitions and bibliography database(s)
%\bibliography{IEEEabrv,../bib/paper}
%
% <OR> manually copy in the resultant .bbl file
% set second argument of \begin to the number of references
% (used to reserve space for the reference number labels box)
\begin{thebibliography}{1}

\bibitem{IEEEhowto:kopka}
H.~Kopka and P.~W. Daly, \emph{A Guide to \LaTeX}, 3rd~ed.\hskip 1em plus
  0.5em minus 0.4em\relax Harlow, England: Addison-Wesley, 1999.

\end{thebibliography}

% biography section
% 
% If you have an EPS/PDF photo (graphicx package needed) extra braces are
% needed around the contents of the optional argument to biography to prevent
% the LaTeX parser from getting confused when it sees the complicated
% \includegraphics command within an optional argument. (You could create
% your own custom macro containing the \includegraphics command to make things
% simpler here.)
%\begin{biography}[{\includegraphics[width=1in,height=1.25in,clip,keepaspectratio]{mshell}}]{Michael Shell}
% or if you just want to reserve a space for a photo:

\begin{IEEEbiography}[{\includegraphics[width=1in,height=1.25in,clip,keepaspectratio]{picture}}]{John Doe}
\blindtext
\end{IEEEbiography}

% You can push biographies down or up by placing
% a \vfill before or after them. The appropriate
% use of \vfill depends on what kind of text is
% on the last page and whether or not the columns
% are being equalized.

%\vfill

% Can be used to pull up biographies so that the bottom of the last one
% is flush with the other column.
%\enlargethispage{-5in}




% that's all folks
\end{document}


